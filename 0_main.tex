% updated April 2002 by Antje Endemann
% Based on CVPR 07 and LNCS, with modifications by DAF, AZ and elle, 2008 and AA, 2010, and CC, 2011; TT, 2014; AAS, 2016; AAS, 2020

\documentclass[runningheads]{llncs}
\usepackage{graphicx}
\usepackage{comment}
\usepackage{amsmath,amssymb} % define this before the line numbering.
\usepackage{color}
\usepackage{cite}
\usepackage{caption} 

\usepackage{subcaption}

\captionsetup[table]{skip=5pt}

% INITIAL SUBMISSION - The following two lines are NOT commented
% CAMERA READY - Comment OUT the following two lines
\usepackage{ruler}
\usepackage[width=122mm,left=12mm,paperwidth=146mm,height=193mm,top=12mm,paperheight=217mm]{geometry}

\usepackage{xcolor}
\usepackage{color,soul}
\definecolor{darkgreen}{rgb}{0.0, 0.8, 0.0}
\DeclareRobustCommand{\cm}[1]{{\begingroup\sethlcolor{darkgreen}\hl{\textbf{comment: 
#1}}\endgroup}}


% ------------------ custom command ------------------ 

\definecolor{c_green}{rgb}{0.0,0.75,0.0}
\definecolor{c_red}{rgb}{1,0.0,0}

\newcommand\MyBox[1]{%
	\fbox{\parbox[c][0.25cm][c]{0.8cm}{\centering #1}}%
}
\newcommand\MyVBox[1]{%
	\parbox[c][0.25cm][c]{0.8cm}{\centering\bfseries #1}%
}  
\newcommand\MyHBox[2][\dimexpr0.8cm+2\fboxsep\relax]{%
	\parbox[c][0.8cm][c]{#1}{\centering\bfseries #2}%
}  
\newcommand\MyTBox[8]{%
	\MyVBox{#1}
	\MyBox{#2}\hspace*{-\fboxrule}%
	\MyBox{#3}\hspace*{-\fboxrule}%
	\MyBox{#4}\hspace*{-\fboxrule}%
	\MyBox{#5}\hspace*{-\fboxrule}%
	\MyBox{#6}\hspace*{-\fboxrule}%
	\MyBox{#7}\hspace*{-\fboxrule}%
	\MyBox{#8}\par\vspace{-\fboxrule}
}  

\newcommand{\AK}[1]{\colorlet{currcol}{.}{\color{orange!80!red}\it[Amir: #1]}\color{currcol}\xspace}
% ------------------ custom command ------------------ 


\begin{document}
% \renewcommand\thelinenumber{\color[rgb]{0.2,0.5,0.8}\normalfont\sffamily\scriptsize\arabic{linenumber}\color[rgb]{0,0,0}}
% \renewcommand\makeLineNumber {\hss\thelinenumber\ \hspace{6mm} \rlap{\hskip\textwidth\ \hspace{6.5mm}\thelinenumber}}
% \linenumbers
\pagestyle{headings}
\mainmatter
\def\ECCVSubNumber{5180}  % Insert your submission number here

\title{A Two-Stage Minimum Cost Multicut Approach \\ to Self-Supervised Multiple
Person Tracking} % Replace with your title

% INITIAL SUBMISSION 
%\begin{comment}
\titlerunning{ECCV-20 submission ID \ECCVSubNumber} 
\authorrunning{ECCV-20 submission ID \ECCVSubNumber} 
\author{Anonymous ECCV submission}
\institute{Paper ID \ECCVSubNumber}
%\end{comment}
%******************

% CAMERA READY SUBMISSION
\begin{comment}
\titlerunning{Abbreviated paper title}
% If the paper title is too long for the running head, you can set
% an abbreviated paper title here
%
\author{First Author\inst{1}\orcidID{0000-1111-2222-3333} \and
Second Author\inst{2,3}\orcidID{1111-2222-3333-4444} \and
Third Author\inst{3}\orcidID{2222--3333-4444-5555}}
%
\authorrunning{F. Author et al.}
% First names are abbreviated in the running head.
% If there are more than two authors, 'et al.' is used.
%
\institute{Princeton University, Princeton NJ 08544, USA \and
Springer Heidelberg, Tiergartenstr. 17, 69121 Heidelberg, Germany
\email{lncs@springer.com}\\
\url{http://www.springer.com/gp/computer-science/lncs} \and
ABC Institute, Rupert-Karls-University Heidelberg, Heidelberg, Germany\\
\email{\{abc,lncs\}@uni-heidelberg.de}}
\end{comment}
%******************
\maketitle

\begin{abstract}
Multiple Object Tracking (MOT) is a long-standing task in computer vision. Current approaches based on the tracking by detection paradigm either require some sort of domain knowledge or supervision to associate data correctly into tracks. In this work, we present a self-supervised multiple object tracking approach based on visual features and minimum cost lifted multicuts. Our method is based on straight-forward spatio-temporal cues that can be extracted from neighboring frames in an image sequences without supervision. Clustering based on these cues enables us to learn the required appearance invariances for the tracking task at hand and train an AutoEncoder to generate suitable latent representations. Thus, the resulting latent representations can serve as robust appearance cues for tracking even over large temporal distances where no reliable spatio-temporal features can be extracted. We show that, despite being trained without using the provided annotations, our model provides competitive results on the challenging MOT Benchmark for pedestrian tracking.

\keywords{Multiple Object Tracking, Self-Supervervised Learning}
\end{abstract}



%%%%%%%%% BODY TEXT
\section{Introduction}

The objective of multiple object tracking is to find a trajectory for each individual object of interest in a given input video. 
Specific interest has been devoted to the specific task of multiple person tracking~\cite{zamir2012gmcp,henschel2017improvements,tang2016multi,tang2017multiple,luo2014multiple}. 
Most successful approaches follow the \textit{Tracking-By-Detection} paradigm.
First, an object (pedestrian) detector is used in order to retrieve the position of each person within each frame. 
Secondly, the output detections of same persons across video frames are associated over space and time in order to form unique trajectories. 
Since objects might get occluded during the video sequence or the detector might simply fail on some examples, successful approaches are usually based not solely on spatial but also on appearance cues. These are learned from annotated data, for example using Siamese networks for person re-identification~\cite{tang2017multiple}.\\

\begin{figure}[t]
	\begin{center}
		%\fbox{\rule{0pt}{2in} \rule{0.9\linewidth}{0pt}}
		\includegraphics[width=0.8\linewidth]{Fig_1_teaser.pdf}
	\end{center}
	\caption{Visualization of the proposed idea: a) Appearance centroid from spatio-temporal information. b) The appearance space is trained to minimize the distance between centroid and associated detections.}
	\label{fig:teaser}
\vspace*{-5mm}
\end{figure}
\noindent\textbf{Motivation.}
Supervised approaches for person re-identification require large amounts of sequence specific data in order to achieve good performance. For this reason, multiple object tracking benchmarks such as MOT~\cite{MOT16} are providing a training  sequence recorded in a sufficiently similar setting for every test sequence. The results of our experiments in table \ref{tab:supervised_compare} confirm this dependency and show the high variance in the quality of supervised approaches, depending on the data used for training. The standard approach to solve this problem is to incorporate additional annotated training data. For example, \cite{yoon2018online, feng2019multi} showed that additional data is key to improving the overall tracking performance.
\begin{table}
	\begin{center}
	{
    \small
		\offinterlineskip
		\hspace*{3mm}
		\hspace*{0.9cm}\MyHBox{02}\MyHBox{04}\MyHBox{05}\MyHBox{09}\MyHBox{10}\MyHBox{11}\MyHBox{13}\par
        \vspace{-0.2cm}
		\MyTBox{02}{\textbf{100}}{\textcolor{c_red}{-0.3}}{\textcolor{c_red}{-0.6}}{\textcolor{c_red}{-11.9}}{\textcolor{c_red}{-11.9}}{\textcolor{c_red}{-9.8}}{\textcolor{c_red}{-14.0}}
		\MyTBox{04}{0.0}{\textbf{100}}{0.0}{\textcolor{c_red}{-11.6}}{\textcolor{c_red}{-11.6}}{\textcolor{c_red}{-4.9}}{\textcolor{c_red}{-9.5}}
		\MyTBox{05}{\textcolor{c_red}{-0.2}}{\textcolor{c_red}{-0.6}}{\textbf{100.0}}{\textcolor{c_red}{-2.0}}{\textcolor{c_red}{-2.0}}{\textcolor{c_red}{-5.1}}{\textcolor{c_red}{-3.6}}
		\MyTBox{09}{0.0}{\textcolor{c_red}{-0.2}}{0.0}{\textbf{100.0}}{0.0}{\textcolor{c_red}{-2.2}}{\textcolor{c_red}{-0.5}}
		\MyTBox{10}{0.0}{\textcolor{c_red}{-0.4}}{\textcolor{c_green}{0.2}}{0.0}{\textbf{100.0}}{\textcolor{c_red}{-0.8}}{\textcolor{c_green}{0.4}}
		\MyTBox{11}{\textcolor{c_green}{0.9}}{\textcolor{c_green}{0.9}}{\textcolor{c_green}{1.0}}{\textcolor{c_green}{1.6}}{\textcolor{c_green}{1.6}}{\textbf{100.0}}{\textcolor{c_green}{1.0}}
		\MyTBox{13}{\textcolor{c_green}{2.2}}{\textcolor{c_green}{2.2}}{\textcolor{c_green}{2.2}}{\textcolor{c_green}{0.2}}{\textcolor{c_green}{0.2}}{\textcolor{c_green}{1.3}}{\textbf{100.0}}
	}
	\vspace*{5mm}
	\caption{Relative MOTA decay for non-matching sequences on MOT17, FRCNN in comparison to the baseline (bold). Columns represent the training sequence, rows the test sequence. 
}
	\vspace*{-5mm}
	\label{tab:supervised_compare}
	\end{center}
\end{table}


Thus, publicly available, \footnotetext{ 
Specifically, we mine GT tracklets from the detections with IoU $>0.5$ with the GT as e.g. done in~\cite{leal2016learning}.}annotated training data currently seems not to be sufficient for training reliable person re-identification networks. Furthermore, recording and labeling sufficient data in a setting close to a final test scenario usually comes at a high price. Hence, the need for methods with a low amount of supervision becomes obvious and motivates us to propose a multiple object tracking method based on self-supervision.\\
While self-supervised learning methods~\cite{kolesnikov2019revisiting} have been successfully exploited in other vision tasks~\cite{watch_move,mahendran2018cross,hendrycks2019using,self_pedest,bb_annot,col_vid}, a direct application to tracking is non-trivial: Learning suitable object appearance metrics for object tracking in a self-supervised way is challenging since, compared to classical clustering problems, visual features of the same person may change over time due to pose and viewpoint changes.
Other issues, such as frequent and long range occlusion or background noises, makes pedestrian tracking even more challenging.\\

%Recently, Deep Neural Networks (DNN) have gained significant role in research and industry. DNN enables ``learning from data'', therefore it is often used when a significant large amount of data is available where the goal is to find patterns or features in it. 
%By extracting these features, one can build a model to correctly classify unseen data if the correct labels are provided. 
%This is called supervised learning. 
%On the other hand, unsupervised approaches group similar data together into clusters.
%This is often more challenging as the model has to associate the correct data together without any ground truth. 
%Self-supervised approaches however use unlabelled data to formulate a pretext task, where the objective can be derived without supervision~\cite{kolesnikov2019revisiting}.

In this paper, we propose an approach for learning appearance features for multiple object tracking without utilizing human annotations of the data. 
Our approach is based on two observations:  I) given an image sequence, many data associations can be made reliably from pure spatio-temporal cues such as the intersection over union (IoU) of bounding boxes within one frame or between neighboring frames. 
II) Resulting tracklets, carry important information about the variation of an object's appearance over time, for example by changes of the pose or viewpoint. 
In our model, we cluster the initial data based on simple spatial cues using the recently successful minimum cost multicut approach \cite{tang2016multi}. 
The resulting clustering information is then injected into a convolutional AutoEncoder to enforce detections with the same, spatio-temporally determined label to be close to one-another in the latent space (see Fig.\ref{fig:teaser}). 
Thus, the resulting latent data representation is encoding not only the pure object appearance, but also the expected appearance variations within one object ID. 
Distances between such latent representations can serve to re-identify objects even after long temporal distances, where no reliable spatio-temporal cues could be extracted. 
We use the resulting information in the minimum cost lifted multicut framework, similar to the formulation of Tang~\cite{tang2017multiple}, whose method is based on Siamese networks trained in a fully supervised way. To summarize, our contributions are:
\begin{itemize}
    \item We present an approach for multiple object tracking, including long range connections between objects, which is completely supervision-free in the sense that no human annotations of the data are employed.
	\item We propose to inject spatio-temporally derived information into convolutional AutoEncoder in order to produce a suitable data embedding space for multiple object tracking.
	\item  We evaluate our approach on the challenging MOT17 benchmark and show competitive results without using actual supervision. 
\end{itemize}

The rest of the paper is structured as follows. 
Section \ref{sec:related} discusses the related work on multiple object tracking. 
Our self-supervised approach on multiple object tracking is explained in Section \ref{subsec:method}. 
In Section \ref{sec:results}, we show the tracking performance of our proposed method in the MOT Benchmark ~\cite{MOT16}. %\footnote{\url{https://motchallenge.net/data/MOT17/}}. 
A conclusion is summarized in Section \ref{sec:conclusion}.




%%%%%%%%% BODY TEXT

\section{Related Work}
\label{sec:related}
\noindent\textbf{Multiple Object Tracking}. 
In Multiple Object Tracking, the most common approach is \textit{Tracking by Detection}. 
The objective is to associate detections of individual persons, which may have spatio or temporal changes in the video. 
Thus re-identification over a long range remains a challenging task.
Multiple object tracking by linking bounding box detections (\emph{tracking by detection}) was studied, e.g., in \cite{Pirsiavash:2011:GOG,Andriyenko2012CVPR,Huang:2008:ROT,AndrilukaCVPR2010,FragkiadakiECCV12,Zamir:2012:GMC,Henschel:2014:EMP,tang14ijcv,Henschel:2014:EMP,DBLP:journals/corr/HenschelLCR17}. 
These works solve the combinatorial problem of linking detections over time via different formulations e.g. via integer linear programming \cite{Shitrit:2011:TMP,wang-et-al-2014}, MAP estimation~\cite{Pirsiavash:2011:GOG}, CRFs \cite{10.1007/978-3-319-16817-3_29}, continuous optimization \cite{Andriyenko2012CVPR} or dominant sets \cite{7503631}. 
In such approaches, the pre-grouping of detections into tracklets or non-maximum suppression are commonly used to reduce the computational costs \cite{Huang:2008:ROT,WojekECCV10,AndrilukaCVPR2010,FragkiadakiECCV12,Zamir:2012:GMC,WojekPAMI2013,Henschel:2014:EMP,tang14ijcv}.
For example Zamir~et~al.~\cite{Zamir:2012:GMC} use generalized minimum clique graphs to generate tracklets as well as the final object trajectories.
Non-maximum suppression also plays a crucial role in disjoint path formulations, such as \cite{networkflow1,networkflow2,Chari2015OnPC}.
In the work of Tang et al. \cite{tang2016multi}, local pairwise features based on DeepMatching are used to solve a multicut problem. 
The affinity measure is invariant to camera motion and thus makes it reliable for short term occlusions. 
An extension of this work is found in \cite{tang2017multiple}, where additional long range information is included. 
By introducing a lifted edge in the graph, an improvement of person re-identification has been achieved.
In \cite{keuper2018motion}, low-level point trajectories and the detections are combined to jointly solve a co-clustering problem, where dependencies are established between the low-level points and the detections. 
Henschel et al. \cite{henschel2018fusion} solves the multiple object tracking problem by incorporating additional head detecion to the full body detection while in \cite{henschel2019multiple}, they use a body and joint detector to improve the quality of the provided noisy detections from the benchmark.
Other works that treat Multiple object tracking as a graph-based problem can be found in \cite{henschel2017improvements}, \cite{keuper2015efficient}, \cite{keuper2016multi}, \cite{kumar2014multiple} and \cite{zamir2012gmcp}.
In contrast, \cite{ma2018trajectory} introduces a tracklet-to-tracklet method based a combination of Deep Neural Networks, called \textit{Deep Siamese Bi-GRU}. 
The visual appearance of detections are extracted with CNNs and RNNs in order to generate a tracklet of individuals.
These tracklets are then split and reconnected in a way that occluded persons are correctly re-identified. 
The framework uses spatial and temporal information from the detector to associate the tracklet. 
The approach in~\cite{bergmann2019tracking} exploits the bounding box information by learning from detectors first and combined with a re-identification model trained on a siamese network. 
While the state of the art approaches in MOT17 Challenge are all based on supervised learning \cite{henschel2018fusion, keuper2018motion, kim2015multiple, 8533372, chen2017enhancing}, there are similar works in \cite{li2018unsupervised, lv2018unsupervised}, which attempt to solve person re-identification (ReID) problems in an unsupervised manner.\\

\noindent\textbf{Self-supervised learning} aims to generate pseudo label automatically from a pretext task, then applied on for unlabeled data in order to solve downstream tasks. 
This is especially useful when no label data is available and thus self-supervised approaches can be applied to many specific real-world problems. An extensive review of recent methods is presented in~\cite{jing2019self}. 
For instance \cite{pathak2017learning} uses a motion-based approach to obtain labels to train a convolutional neural network for semantic segmentation problems.
Another work on self-supervision based on motion can be found in \cite{mahendran2018cross}
The idea of Doersch et al.~\cite{doersch2015unsupervised} is to predict the position of eight spatial configurations given an image pair.
In~\cite{pathak2016context} semantic inpainting task is solved using a context encoder to predict missing pixels of an image.
Hendrycks et al.~\cite{hendrycks2019using} use self-supervised method to improve robustness of deep learning models.
Lee et al.~\cite{lee2019multi} propose an approach to improve of object detection by recycling the
bounding box labels while Ye et al.~\cite{ye2017self} uses are progressive latent model to learn a customized detector based on spatio-temporal proposals.
%%%%%%%%% BODY TEXT
\section{AutoEncoder-Based Multicut Approach}
\label{subsec:method}
The proposed approach is based on the idea to learn, from simple spatial data associations between object detections in image sequences, which appearance variations are to be expected within one object for the task of multiple object tracking \AK{this sentence is not clear to me}. An overview of our workflow implementing this idea is given in Fig.~\ref{fig:figure0}.

\textbf{Stage 1.} 
The object detection bounding boxes are extracted along with their spatial information such that spatial correspondences between detections in neighboring frames can be computed.
Based on these simple spatial associations, detections can be grouped into tracklets in order to obtain cluster labels using clustering approaches such as correlation clustering, also referred to as minimum cost multicuts~\cite{demaine-2006}.

\textbf{Stage 2.} 
A convolutional AutoEncoder is trained to learn the visual features of detections. 
The objective is to learn a latent space representation which can serve to match the same object in different video frames. 
Thus, the information about spatial cluster labels from the first stage is used as the centroid of latent features. 
Distances between latent representations of data samples and their centroids are minimized in the convolutional AutoEncoder using a clustering loss. 

Lastly, the data are transformed into the latent space of the trained AutoEncoder to extract pairwise appearance distances which are expected to encode the desired invariances. 
Such pairwise appearance distances are used to not only provide additional grouping information between nearby detections, but also for detections with long temporal distance. 
The final detection grouping is computed using minimum cost lifted multicuts~\cite{keupericcv}.

This section is divided into three subsections: 
Section \ref{subsec:multicut} describes the minimum cost (lifted) multicut approach employed for obtaining the initial spatial cluster labels (e.g. tracklets), as well as for the generation of the final tracking result. Section \ref{subsec:autoencoder} describes the feature learning process using a convolutional AutoEncoder and cluster labels, and section~\ref{subsec:affinity} describes the computation of the joint spatial and appearance metrics used in the final data association step within the minimum cost lifted multicut framework.

\begin{figure*}[t]
	\begin{center}
		\includegraphics[width=0.9\linewidth]{Fig_2_figure2.pdf}
	\end{center}
	\caption{Summary of our approach in two steps: 1. Initial multicut clustering to obtain weak cluster labels (tracklets), 2. Learn visual features by an AutoEncoder in order to extract affinity information to obtain final clustering (tracking) using lifted multicuts.}
	\label{fig:figure0}
\vspace*{-5mm}
\end{figure*}

\subsection{Multicut Formulation}
\label{subsec:multicut}

We follow Tang~\cite{tang2017multiple} and phrase the multiple target tracking problem as a graph partitioning problem, more concretely, as a minimum cost (lifted) multicut problem. This formulation can serve as well for an initial tracklet generation process, which will help us to inject cues learned from spatial information into the appearance features, as it can be used to generate the final tracking result using short- and long-range information between object detections. \\

\noindent\textbf{Minimum Cost Multicut Problem}.
%%%
We assume, we are given an undirected graph {\it $G = (V, E)$}, where nodes $v\in V$ represent object detections and edges $e\in E$ encode their respective spatio-temporal connectivity. 
Additionally, we are given real valued costs {\it $c: E \rightarrow \mathbb{R}$} defined on all edges. 
Our goal is to determine \emph{edge} labels {\it $y: E \rightarrow \{0, 1\}$} defining a graph decomposition such that every partition of the graph corresponds to exactly one object track (or tracklet). 
To infer such an edge labeling, we can solve instances of the minimum cost multicut problem with respect to the graph G and costs c, defined as follows \cite{chopra-1993,demaine-2006}:

\begin{align}
\min\limits_{y \in \{0, 1\}^E}
\sum\limits_{e \in E} c_e y_e
\label{eq:mc}
\end{align}

\begin{align}
s.t. \quad \forall C \in cycles(G) \quad \forall e \in C : y_e \leq \sum\limits_{e^\prime \in C\backslash\{e\}} y_{e^\prime}
\label{eq:cycle}
\end{align}
Here, the objective is simply to cut those edges with negative costs $c_e$ such that the resulting cut is a decomposition of the graph. 
This condition is formalized by the cycle inequalities in Eq.~\eqref{eq:cycle}, which make sure that, for every cycle in $G$, if one of its edges is cut, so is at least one other.
Thus, no two nodes can remain connected via some path of the graph if an edge is cut between them along any other path. 
In \cite{chopra-1993}, it was shown to be sufficient to enforce Eq.~\eqref{eq:cycle} on all \emph{chordless} cycles, i.e. all cycles.

Typically, if cut probabilities between pairs of nodes are available, the costs are computed using the \emph{logit} function $\text{logit}(p) = \log\frac{p}{1-p}$  to generate positive and negative costs.
With these costs set appropriately, the optimal solution of minimum cost multicut problems not only yields an optimal cluster assignment but also estimates the number of clusters (e.g. objects to track) automatically. 

While the plain minimum cost multicut problem has shown good performance in multiple object tracking scenarios with only short range information available~\cite{tang2016multi}, the cost function actually has a rather limited expressiveness. 
In particular, when we want to add connectivity cues between temporally distant bounding boxes, we can only do so by inserting a direct edge into the graph. 
This facilitates solutions that directly connect such distant nodes even if this link is not justified by any path through space and time. 
This limitation is alleviated by the formulation of minimum cost \emph{lifted} multicuts~\cite{keupericcv}.\\

\noindent\textbf{Minimum Cost Lifted Multicut Problem}.
%%%
For a given, undirected graph $G = (V, E)$ and an additional edge set $F\subseteq \binom{V}{2} \setminus E$ and any  real valued cost function $c: E \cup F \to \mathbb{R}$,
the 01 linear program written below is an instance of the
\emph{Minimum Cost Lifted Multicut Problem (LMP)}
w.r.t.~$G$, $F$ and $c$~\cite{keupericcv}:
%
\begin{equation}
\displaystyle\min_{y \in Y_{EF}} 
    \quad \sum_{e \in E \cup F} c_e y_e
    \label{eq:lmc}
\end{equation}
%
with $Y_{EF} \subseteq \{0,1\}^{E \cup F}$ the set of all $y \in \{0,1\}^{E \cup F}$ with
%
\begin{equation}
 \qquad\forall C \in \textnormal{cycles}(G)\ \forall e \in C:\ 
    y_e \leq \hspace{-2ex} \sum_{e' \in C \setminus \{e\}} \hspace{-2ex} y_{e'} 
\label{eq:lmc-cut1}
\end{equation}
\begin{equation}
 \forall vw \in F\ \forall P \in vw\textnormal{-paths}(G):\ 
    y_{vw} \leq \sum_{e \in P} y_e \quad
\label{eq:lmc-cut2}\end{equation}
\begin{equation}
 \hspace{-1ex} \forall vw \in F\ \forall C \in vw\textnormal{-cuts}(G):
    1 - y_{vw} \leq \sum_{e \in C} (1 - y_e) 
\label{eq:lmc-cut}
   \end{equation}

%$= E\cup E^\prime)$}, real valued costs {\it $c: F \rightarrow \mathbb{R}$} and edge labels {\it $y: F \rightarrow \{0, 1\}$}, the minimum cost \emph{lifted} multicut problem with respect to lifted graph {\it $G$} is defined as follows
The above inequalities Eq.~\eqref{eq:lmc-cut1} makes sure that, as before, the resulting edge labeling is actually inducing a decomposition of $G$. Eq.~\eqref{eq:lmc-cut2} enforces the same constraints on cycles involving edges from $F$, i.e. so called \emph{lifted} edges, and Eq.\eqref{eq:lmc-cut} makes sure that nodes that are connected via a lifted edge $e\in F$ are connected via some path along original edges  $e'\in E$ as well. 
Thus, this formulation allows for a generalization of the cost function to include long range information without altering the set of feasible solutions.\\

\noindent\textbf{Optimization}.
The minimum cost multicut problem~\eqref{eq:mc} as well as the minimum lifted multicut problem~\eqref{eq:lmc} are NP-hard \cite{bansal-2004} and even APX-hard \cite{demaine-2006,hornakova-2017}. 
Nonetheless, instances have been solved within tight bounds, e.g. in \cite{andres-2012-globally} using a branch-and-cut approach. 
While this can be reasonably fast for some, easier problem instances, it can take arbitrarily long for others. 
Thus, primal heuristics such as the one proposed in~\cite{keupericcv} or~\cite{CGC} are often employed in practice and show convincing results in various scenarios \cite{keupericcv,tang2017multiple,insafutdinov2016deepercut,node_agglom}.\\

\noindent\textbf{Spatio-Temporal Tracklet Generation}.
Since the proposed approach is self-supervised in a sense that no annotated labels from the dataset are used in the training process, it is challenging to effectively learn such probabilities. 
To approach this challenge, we first extract reliable point matches between neighboring frames using DeepMatching~\cite{dm} as done before e.g. in \cite{tang2016multi,tang2017multiple,keuper2018motion}. 
Instead of learning a regression model on features derived from the resulting point matches, we simply assume that the intersection over union (IoU) of retrieved matched within pairs of detections (denoted by IoU$_{\mathrm{DM}}$) is an approximation to the true IoU. 
Thus, when IoU$_{\mathrm{DM}}>0.7$, we can be sure we are looking at the same object in different frames. 
While this rough estimation is not suitable in the actual tracking task since it clearly over-estimates the cut probability, it can be used to perform a pre-grouping of detections that definitely belong to the same person. 
The computation of pairwise cut probabilities used in the lifted multicut step for the final tracking task is described in section~\ref{subsec:affinity}.


\subsection{Deep Convolutional AutoEncoder}
\label{subsec:autoencoder}
A convolutional AutoEncoder takes an input image, compresses it into a \textit{latent space} and reconstructs it with the objective to learn meaningful features in an unsupervised manner. 
It consists of two parts: the encoder $f_\theta(.)$ and a decoder $g_\phi(.)$, where $\theta$ and $\phi$ are trainable parameters of the encoder and decoder, respectively. 
For a given input video, there are in total $n$ detections ${x_i \in X}_{i=1}^n$, the objective is to find a meaningful encoding $z_i$, where the dimension of $z_i$ is much lower than $x_i$. The used convolutional AutoEncoder first maps the input data into a latent space $Z$ with a non-linear function $f_\theta: X \rightarrow Z$, then decodes $Z$ to its input with $g_\phi:Z\rightarrow X$. 
The encoding and reconstruction is achieved by minimizing the following loss equation:

\begin{equation}
\label{eq:recloss}
\min_{\theta, \phi} \sum_{i=1}^{N} L (g(f(x_i)), x_i)
\end{equation}

where L is the least-squared loss $L(x,y)=\|x-y\|^2$.
Similar to the work of \cite{yang2016towards}, we add an additional clustering term to minimize the distance between learned features and their cluster center $\tilde{c_i}$ from the spatio-temporal tracklet labels.

\begin{equation}
\label{eq:clusterloss}
\min_{\theta, \phi} \sum_{i=1}^{N} L (g(f(x_i)), x_i)\lambda + L (f(x_i),\tilde{c_i})(1-\lambda)
\end{equation}

The parameter $\lambda \in [0, 1]$ balances between reconstruction and clustering loss.
When choosing $0 < \lambda < 1$, the reconstruction part (Eq.~\eqref{eq:recloss}) can be considered to be a data-dependent regularization for the clustering.
To compute the centroid $c_i$, the whole dataset is passed through the AutoEncoder once:

\begin{equation}
\label{eq:centroid}
\tilde{c_i} = \frac{1}{N} \sum_{i=1}^{N} f(x_i)
\end{equation}

We use a deep AutoEncoder with five convolutional and max-pooling layers for the encoder and five up-convolutional and upsample layers for the decoder, respectively. 
Furthermore, batch normalization is applied on each layer and initialized using Xavier Initialization \cite{glorot2010understanding}. 
The input image size is halved after each layer while the number of filters are doubled. 
The size of latent space is set to 32. 
The input layer takes a colored image with dimension $128\times 128$ in width and height and we applied ReLu activation functions on each layer.

\subsection{AutoEncoder-based Affinity Measure}
\label{subsec:affinity}
We use the trained AutoEncoder to estimate the similarity of two detections $x_i$ and $x_j$ of a video sequence based on the Euclidean distance in the latent space:

\begin{equation}
\label{eq:euclid}
d_{i, j} = \|f(x_i) - f(x_j)\|
\end{equation}

\begin{figure*}[t]
	\begin{center}
		%\fbox{\rule{0pt}{2in} \rule{0.9\linewidth}{0pt}}
		\includegraphics[width=1.0\linewidth]{Fig_4_figure.pdf}
	\end{center}
	\caption{Nearest Neighbor of the query detection (left) within 40 \AK{46 frames} frames with a step size of 5 frames of the sequence MOT17-09-SDP. The second row shows a False Positive neighbor. \AK{this is not clear to me.}}
	\label{fig:figure3}
\end{figure*}

Figure~\ref{fig:figure3} shows the nearest neighbor of a selected frame $t$ (left box marked in red) from the sequence MOT17-09 and frame $t+n$. 
The example illustrates that the location of pair detections are close to one another in the latent space even over a long distance of up to 40 frames, although false positives still appear. 
However, the example also shows that change in appearance affects the AutoEncoder distance, further denoted $d_{\mathrm{AE}}$. 
For instance in the first row, frame 1 and frame 6 are very similar due to the same detection position of the person within the bounding box as well as the direction the woman is looking to (face to camera). 
At frame 41, the woman \AK{isn't that a girl? (in fig. 3)} slightly turned her face to the right. 
Although the correct nearest neighbour was retrieved, the distance $d_{\mathrm{AE}}$ almost doubled.
Another observation is that the position of the bounding box influences the latent space distance.
In the second row, the first detection from the left (frame 5), the detection of the person is slightly shifted to the left. 
At frame 15, 20 or 25, the position is slightly zoomed and $d_{\mathrm{AE}}$ increases \AK{which figure exactly it is referring to?}.

\begin{figure}[!t]
	\begin{center}
		%\fbox{\rule{0pt}{2in} \rule{0.9\linewidth}{0pt}}
		\includegraphics[width=0.9\linewidth]{Fig_5_figure.pdf}
	\end{center}
	\caption{TSNE Visualization of the latent space of the trained auto-encoder for the sequence MOT17-04 FRCNN. The colors represent the assigned person IDs.}
	\label{fig:graph4}
	
\end{figure}

Figure \ref{fig:graph4} shows the TSNE-Visualization \cite{maaten2008visualizing} of the latent space from the sequence MOT17-04-FRCNN. 
Our proposed auto-encoder learned the visual features without supervision. 
The different colors represent the cluster labels. 
As shown in the example circled on the bottom left, similar looking persons are very closed in the latent space: The sitting person in white shirt and the lady, wearing a white shirt (example in bottom left).
The visualization also shows that the same person may change the appearance over time (example on the bottom right). 
In the latent space, the \textit{snake}-like shape may indicate that the viewpoint on a person may have changed over time, causing a continuous appearance change. 
When standing still, the change is minimal, which is also observed in the example on the top right corner.
While for nearby frames, we can compute pairwise cues based on the distance between latent feature representation ($d_{\mathrm{AE}}$), as well as on spatial cues (IoU$_{\mathrm{DM}}$), spatial information can not be used to associate detections over longer temporal distances. 
However, to facilitate the re-identification of objects after being fully or partly occluded, such long-range information is needed.
In these cases, we have to purely rely on the learned latent space distance $d_{\mathrm{AE}}$. 

The distance is directly casted to a binary logistic regression to compute the cut probability of the respective edge in graph $G$. 
The label that is used for the regression comes from the DeepMatching IoU. 
If IoU$_{\mathrm{DM}}(x_i,x_j)<T_{low}$ for a threshold $T_{low}$, $x_i$ and $x_j$ most certainly belong to different objects. 
If IoU$_{\mathrm{DM}}(x_i,x_j)>T_{high}$ for a different threshold $T_{high}$, they are very likely to match. 
Formally, we estimate a probability $p_e \in [0,1]$ between two detections using a feature vector $f^{(e)}$ by regressing the parameters $beta$ \AK{$\beta$?} of a logistic function:

\begin{equation} 
\label{eq:regression} p_e = \frac{1}{1+\exp(-\langle\beta, f^{(e)}\rangle)}
\end{equation} 
Thus, the costs $c_e$ can intuitively be computed by the logit. To robustly estimate these probabilities, we set $T_{low}$ and $T_{high}$ most conservatively to $0.1$ and $0.7$, respectively. 

From this partial, purely spatially induced labeling, we can estimate cut probabilities for all available features combinations, i.e. possible combinations of IoU$_\mathrm{DM}$ and $d_{\mathrm{AE}}$ within nearby frames and only $d_{\mathrm{AE}}$ for distant frames.
%%%%%%%%% BODY TEXT
\section{Experiments and Results}
\label{sec:results}
We evaluated the proposed method on the MOT17 Benchmark~\cite{MOT16} for multiple person tracking. 
The dataset consists of 14 sequences, divided into train and test sets with 7 sequences each. 
For all sequences, three different detection sets are provided, from the detectors SDP\cite{yang2016exploit}, DPM\cite{felzenszwalb2010object} and FRCNN \cite{ren2015faster}, thus yielding 21 sequences in both data splits. 
The settings between the training and testing scenes are very similar such as moving/static camera, place of recording or view angle, such that learning-based methods usually train on the most similar training sequence for every test sequence.
For the evaluation, we use the standard CLEAR MOTA metric~\cite{MOT15}. 
We reported Tracking Accuracy (MOTA), Precision (MOTP), number of identity switches (IDs), mostly tracked trajectories ratio (MT) and mostly lost trajectories (ML).

\begin{table*}[t]
\small
\begin{center}
	\begin{tabular}{c l c r r r r r r r}
		\hline
		No   & Features & Distance & MOTA & MOTP  & IDs & MT & ML & FP & FN \\
		\hline\hline
		1 & IoU$_{\mathrm{DM}}$ & 1-3 & 47.2 & 83.8 & 2,976 & 313 & 659 & 7,633 & 167,109 \\
		2 & $d_{\mathrm{AE}}$ & 1-3 & 35.9 & 84.2 & 4,611 & 128 & 755 & 6,726 & 204,681 \\
		3 & $d_{\mathrm{AE+C}}$ & 1-3 & 46.6 & 83.7 & 2,290 & 302 & 655 & 7,592 & 170,140 \\
		4 & Combined (1+2) & 1-3 & 49.4 & 83.5 & 1,730 & 381 & 593 & 7,536 & 161,057 \\
		5 & Combined (1+3) & 1-3 & 49.4 & 83.4 & 1,713 & 380 & 594 & 7,786 & 161,084  \\
		\hline
		6 & IoU$_{\mathrm{DM}}$ & 1-5 & 47.1 & 83.6 & 2,808 & 318 & 669  & 11,015 & 164,515 \\
		7 & $d_{\mathrm{AE}}$ & 1-5 & 35.1 & 84.0 & 4,361 & 139 & 745  & 8,810 & 174,717 \\
		8 & $d_{\mathrm{AE+C}}$ & 1-5 & 44.8 & 83.6 & 2,585 & 276 & 660 & 8,810 & 174,717 \\
		9 & Combined (6+7) & 1-5 & 49.7 & 83.3 & 1,567 & 389 & 578 & 9,067 & 158,788 \\
		10 & Combined (6+8) & 1-5 & 49.8 & 83.3 & 1,569 & 388 & 580 & 8,869 & 158,715 \\
		\hline
		\textbf{11} & \textbf{Proposed} & \textbf{1-5} & \textbf{49.9} & \textbf{83.3} & \textbf{1,453} & \textbf{394} & \textbf{579} & \textbf{8,929} & \textbf{158,448} \\
		\hline
	\end{tabular}
    \caption{Tracking Performance using different features on MOT17 Training Dataset. The third column refers to the frame distance over which bounding boxes are connected in the graph. $d_{\mathrm{AE}}$ represents the AutoEncoder latent space distance while $d_{\mathrm{AE+C}}$ includes the clustering term, respectively. Our proposed approach includes lifted edges~\cite{tang2017multiple}. Details in supplementary materials.}
\label{tab:mcsetup}
\vspace*{-8mm}
\end{center}
\end{table*}

\begin{table*}
	\small
	\begin{center}
		\begin{tabular}{l c c c c c c c c}
			\hline
			Sequence & Method &\textbf{MOTA} & MOTP & IDs  & MT & ML & FP & FN \\
			\hline
			Tracktorv2\cite{bergmann2019tracking} & Supervised & \textbf{56.3} & \textbf{78.8} & 1,987 & \textbf{21.1} & 35.3 & \textbf{8,866} & 235,449\\
			LSST17\cite{feng2019multi} & Supervised & 54.7 & 75.9 & \textbf{1,243}  & 20.4 & 40.1 & 26,091 & \textbf{228,434}\\
			Tracktor17\cite{bergmann2019tracking} & Supervised & 53.5 & 78.0 & 2,072 & 19.5 & 36.6 & 12,201 & 248,047 \\
			JBNOT\cite{henschel2019multiple} & Supervised & 52.6 & 77.1 & 3,050 & 19.7 & 35.8 & 31,572 & 232,659 \\
			FAMNet\cite{chu2019famnet} & Supervised & 52.0 & 76.5 & 3,072 & 19.1 & \textbf{33.4} & 14,138 & 253,616 \\
			%eTC17\cite{wang2019exploit} & Supervised & 51.9 & 76.3 & 2,288 & \textbf{23.1} & 35.5 & 36,164 & 232,783 \\
			%FWT\cite{henschel2018fusion} & Supervised & \textbf{51.3} & 77.0 & 2,648 & 47.6 & \textbf{21.4} & \textbf{35.2} & 24,101 & 247,921\\
			%jCC\cite{keuper2018motion} & Supervised & 51.2 & 75.9 & 1,802 & 54.5 & 20.9 & 37.0 & 25,937 & 247,822 \\
		    %	MHT\_DAM\_17\cite{kim2015multiple} & Supervised & 50.7 & 77.5 & 2,314 & 47.2 & 20.8 & 36.9 & 22,875 & 252,889\\
			%TLMHT\_17\cite{8533372} & Supervised & 50.6 & \textbf{77.6} & \textbf{1,407} & \textbf{56.5} & 17.6 & 43.4 & 22,213 & 255,030\\
			%EDMT17\cite{chen2017enhancing} & Supverised & 50.0 & 77.3 & 2,264 & 51.3 & 21.6 & 36.3 & 31,279 & \textbf{247,297}\\
			\hline
			\textbf{Ours} & \textbf{Self-supervised} & 48.1 & 76.7 & 2,350 & 17.7 & 39.9 & 16,839 & 273,819\\
			\hline
			
		\end{tabular}
	\caption{Tracking result compared to other methods on the MOT17 dataset. The best performance is marked in bold.}
	\label{tab:mot17compare}
    \vspace*{-10mm}
	\end{center}
\end{table*}
After providing our implementation details, we report an ablation study on the training sequences of MOT17 in section~\ref{subsec:ablationstudy}. 
Our final results are discussed in section~\ref{subsec:mot17benchark}.\\

\noindent\textbf{Implementation Details}.
\label{subsec:experiment}
Our implementation is based on the Tensorflow Deep Learning Framework. 
We use a convolutional AutoEncoder in order to extract features by optimizing the equation~\eqref{eq:clusterloss}.
Thus no pre-training or any other ground truth is required. 
Furthermore, our pre-processing step is only limited to extracting the detections from all sequences and resizing them to the corresponding size of the AutoEncoder input layer. 
Thus the detections from the MOT17 dataset are directly fed to the AutoEncoder. 
For each sequence from the dataset (MOT17-01 to MOT17-14 with the detector SDP, FRCNN and DPM), one individual model is trained with the same setup and training parameters.
However, it is important to note that the number of detections for each individual person varies significantly:
%This is due to the fact that individual pedestrians are captured in a scene over many frames while others %are quickly passing by or simply missed by the detector.
while some pedestrians are staying in the scene for a long time, others are passing by quickly out of the scene. 
This results different cluster sizes.
To balance this, randomized batches of detections are applied during the training, where each batch contains only images from one single frame. 
This way, one iteration of training contains only detections from unique persons. The initial learning rate is set to $\alpha=0.001$ and decays exponentially by a factor of 10 over time. 
The balancing parameter between reconstruction and clustering loss is set to $\lambda = 0$ at the beginning in order to first learn the visual features of the video sequences. 
After five epochs, the cluster information is included in the training, e.g. $\lambda$ is set to $0.95$ to encode the appearance variations from the spatio-temporal clusters into the latent space of the AutoEncoder.\\

\noindent\textbf{From Clusters to Tracklets}. 
To transform detection clusters into actual tracks, we follow the procedure proposed in~\cite{tang2016multi}, i.e. from all detections within one cluster, we select the one with the best detection score pre frame. Clusters containing less than 5 detections are completely removed and gaps in the resulting tracklets are filled using bilinear interpolation. 

\subsection{Ablation Study}
\label{subsec:ablationstudy}

We investigated feature setups in the minimum cost multicut framework. 
The cut probability between pairs of nodes are computed using a logistic regression function. 
Adding new features directly affects the edge cost between pairs thus resulting in different clustering performances. 
Here, we investigate the extent to which our proposed appearance model improves the tracking performance.
%All studies were conducted on the complete MOT17 training dataset. 

\noindent\textbf{Comparison of different setups}.
Table \ref{tab:mcsetup} shows the evaluated setups and the resulting tracking performance scores. %In total, we conducted 11 experiments. 
The column \textit{Features} lists the added features to the logistic regression model. 
The temporal distances over which bounding boxes are connected in the graph are marked in the column \textit{Distances}. 
The tracking accuracy of experiment 1 and 6, which uses IoU$_{\mathrm{DM}}$ only, is 47.2\% and 47.1\%, respectively. 
Experiment 2+3 and 7+8 compare the different AutoEncoder models: the euclidean distance ($d_{\mathrm{AE}}$) from the AutoEncoder latent space is computed in order to estimate the similarity of each pair detections. 
Here, $d_{\mathrm{AE}}$ denotes the latent space distance before adding the clustering loss while $d_{\mathrm{AE+C}}$ denotes the distance after training of the AutoEncoder with the clustering loss, i.e. our proposed appearance method.\\

\noindent\textbf{Best performance with proposed method}.
The benefit of using the clustering loss on the model training is obvious: for both distances (1-3 and 1-5 frames), the performance is significantly higher. 
For distance 1-3, $d_{\mathrm{AE+C}}$ has a tracking accuracy of 46.6 compared to $d_{\mathrm{AE}}$ (35.9) and for distance 1-5, the MOTA scores are 44.8 and 35.1 for $d_{\mathrm{AE+C}}$ and $d_{\mathrm{AE}}$, respectively. 
Although the scores are lower than using IoU$_{\mathrm{DM}}$, combining them both together increases the performance further. 
This is shown in experiment 4+5 and 9+10, where the best score is achieved with in experiment 10 (proposed method).
We also observe that the number of identity switches (IDs) is reduced with our setup.
Finally, we add a long range edges and solve minimum cost lifted multicut problems on $G$. 
The lifted edges are inserted in a sparsely for 10, 20 and 30 frames distances. 
This is empirically studied and will be provided in the supplementary materials.
Our best performance is achieved using the setup of experiment 11 with a MOTA of 49.9\% using all model components.\\

\subsection{Results}
\label{subsec:mot17benchark}

\noindent\textbf{Tracking Performance on test data}. 
Here, we present and discuss our final tracking results on the MOT17 test dataset. 
Compared to the performance on the training dataset, the MOTA score of our proposed approach is slightly lower (Training: 49.9\% vs. Testing: 48.1\%), which is within the observed variance between different sequences, neglecting excessive parameter tuning.
%The only parameter was choosing the lifted edge range, which increased the tracking accuracy on the training data by roughly 0.1\% and reduced the number of IDs by 116.
The best performance is achieved in conjunction with the SDP-detector while the performance on the noisier DPM detections is weaker.\\

\noindent\textbf{Comparison with other tracking approaches}. 
We compare our method with the current top five best reported tracking methods Tracktorv2\cite{bergmann2019tracking}, LSST17\cite{feng2019multi}, Tracktor17\cite{bergmann2019tracking}, JBNOT\cite{henschel2019multiple} and FAMNet\cite{chu2019famnet}.
We consider a tracking method as supervised when ground truth data is used (for example label data for learning a regression function) or if any pre-trained model is included in the approach.
Table \ref{tab:mot17compare} gives an overview of the scores in different metrics that is being evaluated. 
The best on each category is marked in bold. 
Our proposed method is competitive given the fact that no pre-trained model or any other ground truth is employed. 
%Furthermore, the training of an AutoEncoder is straightforward and simple.
Without using any of the provided human annotations and with identical parameter settings for all sequences and detectors, our resulting MOTA scores are shown to be competitive.\\

%\noindent\textbf{Sequence specific training}. 
%To validate the importance of sequence specific training, we ran a small experiment on the MOT17 training dataset: 
%We evaluated models, each trained on one training sequence using \emph{GT annotations} for the tracklet generation, on other training sequences with different viewpoints and resolutions. 
%Specifically, we mine GT tracklets from the detections with IoU $>0.5$ with the GT as e.g. done in~\cite{leal2016learning}.
%Tab. \ref{tab:supervised_compare} shows the results in terms of decay in performance when non-matching sequences are used for training. 
%Moreover, total MOTA scores in all setups are below our reported one (49.9\%.). 
%Despite the availability of annotated data in general, self-supervised approaches are thus important for good performance on previously unseen recording settings.
%and are essential of generalization ability for future problems.

%-------------------------------------------------------------------------
%\begin{table}
%    \vspace*{-4mm}
%
%	\begin{center}{
%    \small
%		\offinterlineskip
%		\hspace*{3mm}
%		\hspace*{0.9cm}\MyHBox{02}\MyHBox{04}\MyHBox{05}\MyHBox{09}\MyHBox{10}\MyHBox{11}\MyHBox{13}\par
%        \vspace{-0.2cm}
%		\MyTBox{02}{\textbf{100}}{\textcolor{c_red}{-0.3}}{\textcolor{c_red}{-0.6}}{\textcolor{c_red}{-11.9}}{\textcolor{c_red}{-11.9}}{\textcolor{c_red}{-9.8}}{\textcolor{c_red}{-14.0}}
%		\MyTBox{04}{0.0}{\textbf{100}}{0.0}{\textcolor{c_red}{-11.6}}{\textcolor{c_red}{-11.6}}{\textcolor{c_red}{-4.9}}{\textcolor{c_red}{-9.5}}
%		\MyTBox{05}{\textcolor{c_red}{-0.2}}{\textcolor{c_red}{-0.6}}{\textbf{100.0}}{\textcolor{c_red}{-2.0}}{\textcolor{c_red}{-2.0}}{\textcolor{c_red}{-5.1}}{\textcolor{c_red}{-3.6}}
%		\MyTBox{09}{0.0}{\textcolor{c_red}{-0.2}}{0.0}{\textbf{100.0}}{0.0}{\textcolor{c_red}{-2.2}}{\textcolor{c_red}{-0.5}}
%	}
    %\vspace*{5mm}
%	\caption{Relative MOTA decay for non-matching sequences on MOT17, FRCNN in comparison to the baseline (bold). Columns represent the training sequence, rows the test sequence. 
%    }
%	\label{tab:supervised_compare}
%	\end{center}
%	\vspace*{-10mm}
%
%\end{table}
%%%%%%%%% BODY TEXT
\section{Conclusion}
\label{sec:conclusion}
In this work, we presented an two stage approach towards tracking of multiple persons without the supervision from human annotations. 
First, we group the data based on their spatial-temporal features to obtain weak clusters (tracklets).
Combining the visual features learned from an AutoEncoder with these tracklets, we are able to automatically create robust appearance cues enabling multiple person tracking over a long distance. The result of our proposed method achieves a tracking accuracy of 48.1\% on the MOT17 benchmark.
To the best of our knowledge, we are the first to propose a fully self-supervised but competitive approach to pedestrian tracking on the MOT benchmarks. 
%Supervised approaches often require a large amount of ground truth annotations to learn models such . 
%Using a pre-trained model for feature extraction may be an alternative but often requires parameter tuning to fit the data into the model. 
%In light of the very diverse scenarios in which pedestrian tracking is to be used in practice, the need for such self-supervised tracking methods is obvious.

\newpage

% ---- Bibliography ----
%
% BibTeX users should specify bibliography style 'splncs04'.
% References will then be sorted and formatted in the correct style.
%
\bibliographystyle{splncs04}
\bibliography{egbib}
\end{document}
