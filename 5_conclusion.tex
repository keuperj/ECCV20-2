%%%%%%%%% BODY TEXT
\section{Conclusion}
\label{sec:conclusion}
In this work, we presented an two stage approach towards tracking of multiple persons without the supervision from human annotations. 
First, we group the data based on their spatial-temporal features to obtain weak clusters (tracklets).
Combining the visual features learned from an AutoEncoder with these tracklets, we are able to automatically create robust appearance cues enabling multiple person tracking over a long distance. The result of our proposed method achieves a tracking accuracy of 48.1\% on the MOT17 benchmark.
To the best of our knowledge, we are the first to propose a fully self-supervised but competitive approach to pedestrian tracking on the MOT benchmarks. 
%Supervised approaches often require a large amount of ground truth annotations to learn models such . 
%Using a pre-trained model for feature extraction may be an alternative but often requires parameter tuning to fit the data into the model. 
%In light of the very diverse scenarios in which pedestrian tracking is to be used in practice, the need for such self-supervised tracking methods is obvious.

\newpage